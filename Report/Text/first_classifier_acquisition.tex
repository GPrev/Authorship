For the first classifier, the information we need are based on the style of writing.

\subsubsection{Features}

For each article, we retrieve some information that will be the features of the classifier. We calculate the number of : 
\begin{itemize}
	\item Punctuation marks (for ' " ; ! ? ; : , - ( ). (We do a count for each mark)
	\item Number of words
	\item The number of subsections
\end{itemize}

For this part, we only used the standards java libraries, in order to manipulate files, strings and chars. 

\subsubsection{Implementation}

The work of the data acquisition is done using two main class : DataRepresentation and TextData.
DataRepresentation is used to parse the article count the features. It uses a hash map of features to check, and a function parse(Article) which compute the parsing and return the features and their relative count, into a TextData object.
TextData is an object that contains all the information about a given article, which allow the classifier to quickly get the data. 
