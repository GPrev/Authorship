\label{lab:data2}

For this second section, we decide to focus on the author's style. In fact, each writer possesses some unique characteristic which is called the authorial or writer invariant in stylometry\footnote{Application of the study of linguistic style}. This later is constant for all texts written by an author and different of other authors. The authorial invariant are based on textual properties belonging to either of four categories : lexical, syntactic, structural and content-specific.
\begin{itemize}
	\item Lexical frequencies : the total of words or characters or the average number of words per sentence.
	\item Syntactic features : the structure of sentences which can be simple, complex, conditional or built in punctuation marks. 
	\item Structural attributes : the organization of the text (drawings, headings, signatures).
	\item Content-specific properties : the recognition of some keywords which have special meaning and are importance for the text.
\end{itemize} 

\subsubsection{The second representation}
For the second representation of the data, we decide to work on Part-of-speech Tagging which is very useful in information retrieval. In our case, we want to attribute an author to an unknown text in function of the frequency of using a part-of-speech. 

\subsubsection{Part-of-speech tagging}
Part-of-speech tagging is the process of assigning a part-of-speech to each word in a sentence as noun, verb, adjective, adverb, article, preposition, pronoun, conjunction and interjection.\\

For examples:

\textbf{Heat} is a noun but also can be a verb.

\textbf{in} is a preposition but also can be a noun or an adverb.

To do POS tagging, we choose a set of tags to assign word.

Here is an example of result of the tagging: 

the/DT higher/JJR minimum/JJ wage/NN sign/NN into/IN law/NN tuesday/NNP will/MD be/VB welcome/JJ relief/NN for/IN million/CD of/IN workers/NNS ./.

\begin{itemize}
	\item DT stands for determiner.
	\item JJR stands for adjective, comparative.
    \item JJ stands for adjective.
    \item NN stands for noun.
    \item VB stands for verb base form.
    \item NNP stands for proper noun singular.
    \item NNS stands for noun plural.
    \item MD stands for modal.
    \item IN stands for preposition or subordinating or conjunction.
    \item CD stands for cardinal number.

\end{itemize}


