The objective of this report is to quickly explain both what we had to do and what we've done for our project in knowledge acquisition. 

\subsection{Authorsip Attribution}

There are two objectives for this project. The first one is to make us understand the knowledge acquisition course better, studying and implementing a software solution, to solve a classical problem of this domain. The second objective is to implement a software which is able to guess the author of a given article. So it has to build a classifier given a training set, and then use it to find the author of a given article from a test set.

\subsection{Organization}

For this project we are divided in groups of 6 students, and we have 4 hours to do all the work. Because we are aware that it is quite a short time to do so, we decided to use these few hours to split the work, and discuss about the directions we were heading to. 
As a first step, we all did some documentary research and studied what is currently existing and may be helpful for the project. After that we met, to chose which of these we may use, and therefore to split the work following the architecture of resulting software.

\subsection{Choice of the language}

Because of the limited time to do the project, we decided to use a language that everybody in the group knows, which makes implementation easy, and can work on several platforms. We chose Java, mainly because we all know it pretty well, which saved us a lot of time. Moreover, there are a lot of libraries on this particular subject implemented in Java.